%% LyX 2.2.3 created this file.  For more info, see http://www.lyx.org/.
%% Do not edit unless you really know what you are doing.
\documentclass[english]{article}
\usepackage[T1]{fontenc}
\usepackage[latin9]{inputenc}
\usepackage{geometry}
\geometry{verbose,tmargin=2cm,bmargin=2cm,lmargin=2cm,rmargin=2cm}
\setlength{\parskip}{6pt}
\setlength{\parindent}{0pt}
\usepackage{babel}
\usepackage{url}
\usepackage{amsmath}
\usepackage{amssymb}
\usepackage{graphicx}
\usepackage[authoryear]{natbib}
\usepackage[unicode=true]
 {hyperref}

\makeatletter

%%%%%%%%%%%%%%%%%%%%%%%%%%%%%% LyX specific LaTeX commands.
%% Special footnote code from the package 'stblftnt.sty'
%% Author: Robin Fairbairns -- Last revised Dec 13 1996
\let\SF@@footnote\footnote
\def\footnote{\ifx\protect\@typeset@protect
    \expandafter\SF@@footnote
  \else
    \expandafter\SF@gobble@opt
  \fi
}
\expandafter\def\csname SF@gobble@opt \endcsname{\@ifnextchar[%]
  \SF@gobble@twobracket
  \@gobble
}
\edef\SF@gobble@opt{\noexpand\protect
  \expandafter\noexpand\csname SF@gobble@opt \endcsname}
\def\SF@gobble@twobracket[#1]#2{}
%% A simple dot to overcome graphicx limitations
\newcommand{\lyxdot}{.}


%%%%%%%%%%%%%%%%%%%%%%%%%%%%%% User specified LaTeX commands.
\sloppy
%\setlength{\parskip}{6pt}

\@ifundefined{showcaptionsetup}{}{%
 \PassOptionsToPackage{caption=false}{subfig}}
\usepackage{subfig}
\makeatother

\begin{document}
\global\long\def\taxonomy{\mbox{\ensuremath{\mathbb{T}}}}

\global\long\def\prunedTaxonomy{\taxonomy_{P}}

\global\long\def\phyloinputs{\mathcal{T}}

\global\long\def\expandedPhylo{\phyloinputs_{E}}

\global\long\def\summaryTree{\mathbb{S}}

\global\long\def\prunedSummary{\summaryTree_{P}}

\global\long\def\collections{\mathcal{C}}

\title{\texttt{Handling }\texttt{\emph{incertae sedis}}\texttt{ taxa in
propinquity}}
\maketitle

\section{Introduction}

The phrase \textquotedblleft \emph{incertae sedis}\textquotedblright ,
which is a Latin phrase meaning ``uncertain seat'', is used in classifications
to indicate that a group is of uncertain taxonomic placement. Such
groups arise in practice when they are classified within a parent
group, such as Fungi, but the precise location within the parent group
is uncertain. Such a taxon would then be attached as a child of the
parent group, but labeled as \emph{incertae sedis}''. 

Currently \emph{incertae sedis} taxa and their descendants are excluded
from the OpenTree synthetic tree. This is problematic, because the
OpenTree project seeks to build a comprehensive tree covering all
of life. However, including unplaced taxa within the current synthesis
framework would be equally problematic, since placement of taxa within
groups is currently considered to be a conflict with those groups.
This conflict causes both a loss of the taxon name for the conflicting
taxon, and a loss of structure in the synthesis tree. 

\subsection{Some examples in OTT\protect\footnote{This paragraph pulled from intro to otcetera/doc/handling-incertae-sedis.pdf}}

The problem is becoming more acute because NCBI is putting a larger
number of taxa into groups that are marked as \textquotedblleft unclassified.\textquotedblright{}
For example, when OTT 2.9 was created NCBI\textquoteright s clasification
of the bird family Sylvidae included a group five genera that were
placed within \textquotedblleft \href{http://www.ncbi.nlm.nih.gov/Taxonomy/Browser/wwwtax.cgi?mode=Undef&id=36270&lvl=3&lin=f&keep=1&srchmode=1&unlock}{unclassified Sylviidae}.\textquotedblright{}
This includes the genus Regulus. Thus in OTT 2.9 Regulus (ott ID =
3599326) is placed inside the family Silvidae (OTT ID = 259942). The
\textquotedblleft unclassified Sylviidae\textquotedblright{} does
not appear in OTT; instead Regulus is flagged as \textquotedblleft unclassified,sibling
higher\textquotedblright{} and all of the species within Regulus are
flagged as \textquotedblleft unclassified inherited.\textquotedblright{}

The taxonomy (as of version 2.9) also contains 685 cases of taxa that
are flagged as both \textquotedblleft unclassified inherited\textquotedblright{}
and \textquotedblleft unclassified\textquotedblright \footnote{based on \texttt{grep unclass.{*}unclassified\textbackslash{}\_inherited
taxonomy.tsv}}.

We would like to stop suppressing (pruning) incertae sedis taxa, so
that groups such as Regulus can appear appear in the synthetic tree.

\subsection{Handling incertae sedis taxa}

To solve this problem, we describe an extension of our previous supertree
synthesis procedure that can use phylogeny information to place \emph{incertae
sedis} taxa into containing taxa without considering the containing
taxa to be broken. This extension requires that we first describe
an extended semantics for taxa in the presences of \emph{incertae
sedis}, which is not a trivial problem. We describe both an updated
semantics for conflict with taxonomy edges, and an updated semantics
for assigning taxonomy names to nodes in the synthetic tree. 

The ability to place taxa within the taxonomy without breaking their
containing taxa enables us to include thousands of new taxa that were
previously filtered out to avoid broken taxa. It also enables us to
use phylogenetic trees to update the taxonomy by extending taxa to
include \emph{incertae sedis} taxa placed within them. This makes
substantial progress towards our goal of complete representation of
taxa. It also enables us to include extinct taxa, since many of these
taxa are incertae sedis.

\subsection{Incertae sedis taxa in taxonomies}

\textbf{Flags.} The Open Tree\textquoteright s reference taxonomy
is produced by a tool called smasher that notices hints that a taxon
is incertae sedis and labels that taxon with one of five flags\footnote{see \url{https://github.com/OpenTreeOfLife/reference-taxonomy/wiki/Taxon-flags}.This
document is not concerned with the flagging system \emph{per se},
so \textquotedblleft \emph{incertae sedis}\textquotedblright{} will
be used here to refer to all of the flags that denote taxa with uncertain
placement.}. While a taxonomist may intend to use incertae sedis to indicate
a limited number of possible positions for a taxon to go in the taxonomy,
we do not retain any such details.

\textbf{Container nodes.} Some taxonomies have nodes with names like
``Incertae sedis (Bacteria)''. Such nodes are called containers.
They indicate that each child node is an \emph{incertae sedis} child
of the container node's parent. The OTT retains container nodes with
the \emph{was\_container} flag, but seems to have already moved all
of the container's children to the container's parent.

\textbf{Taxonomy merging.} Not all \emph{incertae sedis} taxa in our
taxonomy are directly labeled \emph{incertae sedis} by taxonomists.
\emph{Incertae sedis} taxa can also result from automatic merging
of taxonomies to create the OpenTree taxonomy. For example, in Figure
3 of \citet{rees2017automated}, cases \#4 and \#6 illustrate examples
where merging of two taxonomies leads to a taxonomy with a taxon of
uncertain placement. The reason is primarily that if taxonomy $\taxonomy_{1}$
contains more levels of hierarchy than taxonomy $\taxonomy_{2}$,
then we must add internal nodes to $\taxonomy_{2}$ to align it to
$\taxonomy_{1}$. However, if taxonomy $\taxonomy_{2}$ contains more
leaves than $\taxonomy_{1}$, then it is unclear if these extra leaves
should be nested inside the additional internal nodes, or not. Thus,
the extra leaves are marked \emph{incertae sedis}.

For example, if $\taxonomy_{1}=((a,b)x,(c,d)y)z$ and $\taxonomy_{2}=(a,b,c,d,e)z$
then $a$ and $b$ in $\taxonomy_{2}$ should be nested within $x$,
but we do not know if $e$ should be nested within $x$ or not. Thus
we obtain $((a,b)x,(c,d)y,?e)z$, where $?$ indicates that taxon
$e$ is marked \emph{incertae sedis}.

\subsection{Formalism}

We consider a ranked collection $\phyloinputs=\left\{ T_{1},\ldots,T_{n}\right\} $
of input trees and a single taxonomy tree $\taxonomy$. We have a
set of labels $\mathcal{L}$ that correspond to taxon names. Therefore
taxonomy node has a label, and every label correspond to a unique
taxonomy node.

In contrast, input phylogenies only have labels on their leaf nodes.
Furthermore, and the set of labels $\mathcal{L}(i)$ on input tree
$T_{i}$ need not include all leaf labels, and in practice always
contains only a relatively small part of $\mathcal{L}$.

We extend this framework by adding a set $\mathcal{I}\subseteq\mathcal{L}$
of labels that have the \emph{incertae sedis }property. The taxonomy
consists of the labels $\mathcal{L}$, the taxonomy tree $\taxonomy$,
and the \emph{incertae sedis} property $\mathcal{I}$ of taxon names.

\section{\label{sec:Semantics-of-incertae}Semantics of \emph{incertae sedis}
taxa}

Incertae sedis taxa affect the synthesis procedure in two ways: they
affect what it means for a taxonomy edge to conflict with the synthesis
tree, and they affect what it means to place a taxon name on a particular
node in the synthesis tree. We examine each of these effects in turn.

\subsection{Splits}

Each edge of a standard tree divides the tip taxa $\mathcal{L}$ into
two groups: the include set $\mathcal{I}(e)$ which does not contain
the root, and the exclude set $\mathcal{E}(e)$ which does contain
the root. Such a split may be written
\[
\mathcal{I}(e)|\bullet\mathcal{E}(e),
\]
with the exclude set always on the right. If no taxa are \emph{incertae
sedis}, then the exclude set for a node is just the total tip set
minus the include set for the node:
\begin{align*}
\mathcal{E}(n) & =\mathcal{L}-\mathcal{I}(n).
\end{align*}

For a node $n$ on the tipward side of an edge $e$, we may also write
$\mathcal{I}(n)$ for $\mathcal{I}(e)$, and $\mathcal{E}(n)$ for
$\mathcal{E}(e)$. We consider the exclude set of the root node to
be empty, and the include set of the root node to contain all tip
taxa:
\begin{align*}
\mathcal{E}(root) & =\{\}\\
\mathcal{I}(root) & =\mathcal{L}.
\end{align*}
In this case the root node stands for an edge that connects the root
to the root's parent.

\subsubsection{Reduced exclude sets for \emph{incertae sedis} taxa}

Marking a taxon as \emph{incertae sedis} changes the meaning of its
sibling taxa. Since the \emph{incertae sedis} taxon can be placed
inside its sibling subtrees, the exclude sets of the sibling subtrees
should not include the \emph{incertae sedis }taxon. Although exclude
sets of sibling taxa are shrunk, the include sets of the sibling taxa
are unaffected.

An \emph{incertae sedis} taxon can be moved into any of the descendants
of its siblings. Therefore, the exclude $\mathcal{E}(n)$ set for
a node $n$ contains the children of all the ancestors of $n$, unless
those children are incertae sedis. We can compute exclude sets recursively
by writing the exclude set of a node $n$ in terms of the exclude
set of its parent. If we use the terminology that the include and
exclude sets for a node $n$ are $\mathcal{I}(n)$ and $\mathcal{E}(n)$,
then we have 
\begin{align}
\mathcal{E}(n) & =\mathcal{E}(parent(n))\cup\left[\mathcal{I}(m)\big|m\in siblings(n),m\text{ not \emph{incertae} \emph{sedis}}\right],\label{eq:exclude-set-formula}
\end{align}
This leads to a pre-order recursion that terminates if the exclude
set for the root note is set to $\{\}$ as a boundary condition.

\subsubsection{Alternative formulation}

Consider an I.S. clade $A$ that is more rootward than an I.S. clade
$B$. With the above formulation, we could not place $A$ as sister
to $B$ and then place $B$ within $A$, because $A$ excludes $B$.
To solve this problem, we could modify formula (\ref{eq:exclude-set-formula})
above to avoid excluding \texttt{incertae\_sedis\_inherited} tips
of a sister taxon $m$. Thus, we could modify formula (\ref{eq:exclude-set-formula})
to refer to $\mathcal{I}^{\prime}(m)$ instead of $\mathcal{I}(m)$,
where 
\begin{align*}
\mathcal{I}^{\prime}(m) & =\begin{cases}
m & \text{if }m\text{ is a leaf}\\
\left[\mathcal{I}^{\prime}(c)\big|c\in children(m),c\text{ not \emph{incertae sedis}}\right] & \text{otherwise.}
\end{cases}
\end{align*}
This would not rule out placing $B$ as sister to $A$ and then placing
$A$ within $B$.

\subsection{Naming }

After solving a supertree (sub) problem, we need to assign taxon names
to the supertree nodes based on the taxonomy tree in the problem.
Each taxon name $n$ corresponds to a split $S(n)=S(n)_{1}|S(n)_{2}$
on the corresponding branch of the taxonomy tree. Without \emph{incertae
sedis}, such splits are always of the form $S(n)_{1}|\mathcal{L}-S(n)_{1}$,
but with \emph{incertae sedis} taxa $S_{2}(n)$ may be smaller than
$\mathcal{L}-S(n)_{1}$.

Without \emph{incertae sedis}, each name applies to at most one node,
and each node can take at most one name, with the exception of monotypic
taxa. Thus, we may simply search the solution tree for a node that
has the same cluster $S(n)_{1}$ and apply the name $n$ to that node.

However, in the \emph{incertae sedis }framework we must raise the
question of whether one name could apply to multiple nodes, or whether
multiple names could apply to one node. 

\textbf{BDR:} \emph{It is well known that monotypic taxa are indistinguishable
if you consider only splits on leaf labels, but are distinguishable
if you consider splits on all node labels. It seems that some (all?)
of the problems with assigning multiple names to the same nodes actually
comes from the fact that moving IS taxa can leave the parent as a
degree-2 node. The fact that propinquity handles monotypic taxa indicates
that we actually implicitly consider all taxa to have labels. It is
possible that a trivial extension to the sub-problem solved could
thus handle monotypic taxa and issues with mapping 2 names to one
node with incertae sedis taxa.}

\subsubsection{Multiple nodes that fit one name}

Suppose the taxonomy is (((A1,A2)A,B)AB,C,D{*}) and the solution tree
is (((A1,A2)x,D{*})y,B)AB,C). In this case the name $A$ leads to
the split S(A) = A1 A2 | B C root. This name can apply to both the
node x and the node y.

\textbf{Solution:} In this case, we find the most tipward node and
attach the name to this node.

\subsubsection{Multiple names fit a single node}

\paragraph{Example 1}

Suppose the taxonomy is (((B1,B2)B,C{*})A,Y) and the input tree is
(((B1,C),B2)x,Y) then the names A and B both to the node x. In this
case the names A and B are ordered.

\textbf{Case 1:} If a taxon contains 2 non-IS taxa, then it cannot
be identical with any of its children in the synthesis tree.

\textbf{Case 2:} If a taxon contains 1 non-IS taxon and $\ge1$ IS
taxa, then the taxon could be identical with is non-IS child in the
synthesis tree, if the IS taxa are placed within the child.

\textbf{Case 3:} If a node contains 0 non-IS taxa and 1 IS taxon,
then the IS taxa behaves no differently than a non-IS taxa, since
is has no siblings it could be placed into.

\textbf{Case 4:} If a node contains 0 non-IS taxa and $\ge2$ IS taxa,
then then taxon \emph{could} be identical with a non-IS child in the
synthesis tree, if all but one IS children are placed with in one
of the IS children.

\textbf{Solution:} When we assign multiple names to the same node,
then we expand the node with multiple names to have monotypic parents,
and assign the series of names to the monotypic parents. Another way
of saying this is that when a node has $\le1$ non-IS taxon and $\ge2$
taxa then the node could become monotypic by placement of the IS taxa.

\textbf{Example 2}

If ((A1,A2)A,(B1,B2)B{*},(C1,C2)C{*}); is the taxonomy with asterisks
denoting incertae sedis taxa, then the solution ((A1,A2)A,((B1,C1)mrcaB1C1,(B2,C2)mrcaB2C2)x);
has a node x that could be called B{*} or C{*}.

\textbf{Case:} If a taxa B{*} and C{*} are IS, then their tips can
be intermingled in a new clade x. Both names would then apply.

\textbf{Solution:} ??

In summary, a taxon with split $A_{1}|\bullet B_{1}$ attaches to
the most tipward node $n$ where $A_{1}\subseteq S_{1}(n)$ and $B_{1}\subseteq S_{2}(n)$.
If multiple names end up on the same node, we try to resolve the problem
by creating a monotypic node for a name that is more rootward than
all other names at that node. If this fails to resolve the problem,
we arbitrarily choose one of the names. 

\section{Synthesis and conflict resolution with incertae sedis taxa}

\subsection{Placement causes broken taxa}

Synthesis with \emph{incertae sedis} taxa has the potential to resolve
uncertain taxon placements using information from phylogenies. The
propinquity pipeline has always been able to do this kind of resolution.
However, without special consideration given to \emph{naming}, placing
a taxon $A$ within a taxon $B$ results in conflict with taxon $B$
in the taxonomy. In order to avoid a situation where input phylogenies
conflict with a large number of taxa when \emph{incertae sedis} taxa
are placed within them, we have filtered \emph{incertae sedis} taxa
from the taxonomy when constructing all prior synthesis trees.

When the synthesis tree conflicts with a taxonomy node, we way that
the taxon $B$ at that node is a broken taxon. Broken taxa have two
main effects, both of which are negative. First, the name $B$ of
the broken taxon is removed from the synthesis tree. Second, the conflicting
edge for taxon $B$ is not included in the synthesis tree. This means
that any children of $B$ that are taxonomy-only will not be placed
with the children of $B$ that are mentioned in input phylogenies.
Instead the taxonomy-only children of a broken taxon move towards
the root of the synthesis tree and attach at the first higher-ranked
taxon that is an ancestor of $B$ but is not broken.

\subsection{Correctly handling placement}

Thus, we seek a synthesis method that can correctly place taxa within
containing taxa without breaking the containing taxa. We change the
semantics of names to imply, not the exclusion of all non-included
taxa, but only some non-included taxa, as described in section \ref{sec:Semantics-of-incertae}.
As a result of this change in semantics, placing an IS taxon $A$
within a sister taxon $B$ no longer results in conflict with $B$.
This allows us to retain the split for $B$ within the synthesis tree,
so that taxonomy-only children of $B$ are correctly grouped with
their siblings that are referenced by the input trees. We may then
retain the name for the no-longer-broken taxon. Finally, we are then
able to stop filtering \emph{incertae sedis} taxa, so that they appear
in the synthesis tree. Thus, the synthesis tree is able to represent
substantially more species, without suffering the loss of taxa and
the loss of structure.

\subsection{Conflict with incertae sedis taxa}

\subsubsection{Conflicting placement among input trees}

The addition of \emph{incertae sedis} taxa allows new types of conflict
between input trees. For example, different input trees might place
an incertae sedis taxon in conflicting locations. This is illustrated
in Figure \ref{fig:placement-example1}, where the IS taxon (ott7,ott6)ott10
is placed as sister to ott1 by phylogeny $\tau_{1}$ and as sister
ott3 by phylogeny $\tau_{2}$. 

When this happens, the placement of the IS taxon is not influenced
by its being marked IS on the taxonomy. Thus, in Example 1, the higher
ranked tree $\tau_{1}$ will be reflected in the synthesis tree, ott10
will be placed as sister to ott1. In contrast, the conflicting placement
in $\tau_{2}$ will not be reflected in the synth tree. 

All this would occur in the previous version of propinquity. Where
the updated version differs that (a) the names ott9 and ott8 are retained
instead of being dropped. (b) as a result of not breaking ott8 and
ott9, we do not move ott3 and ott2 up to ott11. {[}BDR: Extend more
figures!{]}

\subsubsection{Example B}

However, incertae sedis taxa are not always tip nodes, but may themselves
contain other taxa. In such cases, it is possible for input trees
to conflict with the incertae sedis taxon itself. For example, consider
the following example
\begin{itemize}
\item $T_{1}=((w_{1},w_{2},z_{1}),y_{1})$
\item $T_{2}=((y_{1},y_{2},z_{2}),w_{1})$
\item Taxonomy tree $((w_{1},w_{2})w,(y_{1},y_{2})y,(z_{1},z_{2},z_{3})z)$
with $z$ marked \emph{incertae sedis}.
\begin{itemize}
\item splits $w_{1}w_{2}|\bullet y_{1}y_{2}$, $y_{1}y_{2}|\bullet w_{1}w_{2}$
and $z_{1}z_{2}|\bullet w_{1}w_{2}y_{1}y_{2}$
\end{itemize}
\end{itemize}
In this case, tree $T_{1}$ places $z_{1}$ within $w$, while $T_{2}$
places $z_{2}$ within $y$. Since different members of $z$ are placed,
neither placement for $z$ is rejected. Instead the taxon $z$ is
broken, the name $z$ disappears, and $z_{3}$ floats to the top level. 

Furthermore, since taxon $z$ is a broken \emph{incertae sedis }taxon,
all of its children are effectively incertae sedis independently,
with the difference that they cannot be placed within each other.
Therefore, the names $w$ and $y$ are not lost, since the taxa placed
within them are \emph{incertae sedis} names. Note that $z_{3}$ maybe
therefore be placed in a third location, since it is also \emph{incertae
sedis}.

The situation here would be different if the monophyly of $z$ was
supported by an input phylogeny.

\emph{Note that the synthesis of all input trees before the taxonomy
is unaffected by incertae sedis information.}

\subsubsection{Example C - Nested \emph{incertae sedis} taxa}

In addition to incertae sedis taxa containing other taxa, it is also
possible for incertae sedis taxa to contain nested \emph{incertae
sedis} taxa. \textbf{What issues might this raise?}

\subsection{Placement}

Placement of incertae sedis taxa by input trees is unfortunately not
quite as simple as finding a single location where an I.S. taxon should
attach. For example, when an incertae sedis taxon is broken, its children
need to be ``placed'' separately.

Each input tree can relate to an \emph{incertae sedis} taxon $(A,B,C)D$
in a number of ways
\begin{itemize}
\item it could resolve $A$, $B$, $C$, or $D$.
\item it could place $D$ on a degree-2 (=out-degree-1) node that bisects
a branch
\item it could place a descendant taxon of $D$ 
\item it could place a descendant taxon of $D$ in a \emph{different place}
than another input tree.
\item it could place children of $D$ in multiple places, thus conflicting
with the branch. If the \emph{incertae sedis} taxon $(A,B,C)$ is
broken, then $A$, $B$ and $C$ become \emph{incertae sedis} clades
in their own right, that may attach separately, except that they .
This is because none of $A$, $B$, or $C$ is in the exclude set
of the siblings of $D$.
\end{itemize}

\section{Handling \emph{incertae sedis} taxa in the propinquity pipeline}

In order to handle \emph{incertae sedis} taxa within propinquity,
we must modify some of the stages of the propinquity pipeline. Subproblem
decomposition must place \emph{incertae sedis} taxa in the correct
subproblem. Subproblem files must indicate which taxa are incertae
sedis. The subproblem solver must read this information, account for
\emph{incertae sedis} taxa when solving subproblems, and correctly
name taxa that have been modified by having \emph{incertae sedis}
taxa place inside them. The unpruner must be aware of \emph{incertae
sedis} taxa. Annotations of the tree must be aware of \emph{incertae
sedis} taxa so that it does not consider taxa broken when they have
an incertae sedis taxon placed inside them.

\subsection{Exemplifying taxa}

One current problem is that well-known taxa like Fungi or Mammalia
tend to have a very large number of incertae sedis children, making
browsing in the tree viewer difficult. This can happen when, for example,
fossils or other hard-to-place taxa get classified only to the level
of these well-known nodes and no further. This leads to a situation
where well-known taxa serve as a dumping ground for unplaced taxa.

Our current approach to this problem is to perform a second round
of pruning, or ``cleaning'', during the exemplification step. Incertae
sedis taxa are pruned at this stage if they do not occur in any input
trees. We thus generate a second ``cleaned taxonomy'' that has undergone
this further round of cleaning. This approach improves on the previous
approach in that \emph{incertae sedis} taxa in input trees are no
longer pruned. This approach also removes tons of \emph{incertae sedis}
children from nodes like ``Fungi'', where a lot of unplaced fossils
with few observable characters have been dumped. 

However, this approach has the negative effect of pruning some incertae
sedis taxa that need not be pruned. For example, suppose \emph{incertae
sedis} taxon $A$ contains 5 children, of which only 1 child $A_{1}$
occurs in an input tree. If the taxon $A$ is not broken, then it
should be possible to attach the other 4 members of $A$ next to $A_{1}$,
without cluttering up the synthesis tree. Such taxa have been successfully
placed even though they are not in any input tree. This can only be
discovered after synthesis is complete, though.

Additionally, it should also be possible to filter unplaced taxa in
the tree viewer instead of in the synthesis pipeline.

\subsection{Sub-problem decomposition}

The presence of \emph{incertae sedis }taxa poses a problem to sub-problem
decomposition, since taxonomy edges no longer completely separate
subproblems. Instead, \emph{incertae sedis} taxa may attach on either
side of a taxonomy edge. We seek to place \emph{incertae sedis} taxa
into subproblems in such a way that the subproblem solver can perform
the placement inside the subproblem. This approach postpones handling
of conflict in \emph{incertae sedis} taxa to the subproblem solver,
where the problem is well formulated in terms of splits. However,
it does have the effect of creating larger subproblems.

We must also handle conflicting placements of \emph{incertae sedis}
taxa by different input trees. Thus, if one input tree places the
\emph{incertae sedis} taxon $X$ in $((X)B)A$ and another places
$X$ in $((X)C)A$ then we must mark both edges $B$ and $C$ as contested
edges, even if these edges would \emph{not} be contested were taxon
$X$ to be removed. This results in a new way to contest edges that
involves the interaction of two input trees, and not just the interaction
of each input tree with the taxonomy.

We choose to solve these problems by merging any subproblems that
an \emph{incertae sedis} taxon might be placed in. The simplest way
to achieve this is simply to regard any taxon that has an \emph{incertae
sedis }taxon placed within it as contested. This results in marking
both $B$ and $C$ as contested edges in the example above. In fact,
this is the current behaviour of the non-\emph{incertae-sedis} aware
subproblem decomposer. One downside of this approach is that, if we
have $((X)B)A)$ in one input tree, and $X$ is mentioned nowhere
else, then by marking $B$ as contested, we are merging subproblems
unnecessarily. We could instead placed $X$ in $B$ and avoid contesting
the edge $B$. However, this approach is more complex and does not
seem necessary in practice.

\begin{figure}
\subfloat[Taxonomy $\protect\taxonomy$]{\includegraphics[scale=0.5]{Figures/placement1/tax\lyxdot tre}

}\subfloat[$\tau_{1}$]{\includegraphics[scale=0.5]{Figures/placement1/input1\lyxdot tre}

}\subfloat[$\tau_{2}$]{\includegraphics[scale=0.5]{Figures/placement1/input2\lyxdot tre}

}

\caption{\label{fig:placement-example1}Example. An incertae sedis clade (ott6,ott7)
is placed in different subtrees by input trees $\tau_{1}$ and $\tau_{2}$.
In $\tau_{1}$, two nodes that correspond to the taxonomy their ingroup
extended to include (ott6,ott7), and the branches leading to these
nodes have been colored blue. The dashed blue edge leads to a node
that is a newly-introduced degree-2 node which does not correspond
to any taxonomy node. In $\tau_{2}$, only one node that corresponds
to a taxonomy node needs to have its ingroup extended. The placement
of (ott6,ott7) into ott8 toward ott1 by $\tau_{1}$ conflicts with
the placement of (ott6,ott7) into ott9 toward ott3 by $\tau_{2},$}
\end{figure}
 For example, in Figure \ref{fig:placement-example1}, input tree
$\tau_{1}$ contests ott9 and input tree $\tau_{2}$ contests ott9
and ott8. Thus ott1, ott2, ott3, ott6, and ott7 end up in the same
sub-problem.

\subsection{Subproblem solution}

Our sub-problem solver naturally handles \emph{incertae sedis} taxa.
This is because we define the semantics of \emph{incertae sedis} taxa
in terms of partial splits, and our solver natively supports building
trees from partial splits through its use of the BUILD algorithm.
Handling \emph{incertae sedis} taxa thus requires loading incertae
sedis information and computing partial splits for \emph{incertae
sedis} taxa before solving a sub-problem. After solving a sub-problem,
we must apply taxon names from the taxonomy tree to the sub-problem
solution tree. The solution tree is considered to a fixed tree and
not to have any \emph{incertae sedis} nodes, or any other forms of
uncertainty.

\subsubsection{Reading incertae sedis information}

Currently, we read the \emph{incertae sedis} information as a list
of OTT ids for \emph{incertae sedis} taxa. This does not require adding
further annotations to the node names. Only taxonomy nodes can be
\emph{incertae sedis} at the moment, and only the taxonomy tree for
the subproblem contains OTT ids for internal nodes. Therefore we handle
\emph{incertae sedis} information by constructing modified split sets
for the lowest-ranked tree when the list of \emph{incertae sedis}
nodes is not empty.

\subsubsection{Exclude sets modified by \emph{incertae sedis} marks}

Equation (\ref{eq:exclude-set-formula}) leads to the following algorithm
to compute the exclude set for all nodes in a tree.
\begin{enumerate}
\item Set the exclude set of the root node to be empty
\item For each \emph{node} (except the root) in preorder
\begin{itemize}
\item combine the \emph{exclude} set of the parent node with the \emph{include}
set of non-\emph{incertae-sedis} siblings.
\item store this set in a hash, with key \emph{node}
\end{itemize}
\end{enumerate}
This algorithm is currently implemented in \emph{otc-solve-subproblem}.
We store the sets as \emph{std::set}.

\subsubsection{Implementation: finding the node for a name}

To find the node for a name $n$, we find the MRCA of the cluster
$S_{1}(n)$. If the MRCA excludes the entire exclude group $S_{2}(n)$
then the name applies to the MRCA; otherwise the taxon does not exist
on the tree.

\subsubsection{Implementation: handling name clashes}

When multiple names $N=\{n_{1},\ldots n_{N}\}$ map to the same solution
node $x$, then these names must satisfy some tree structure on the
taxonomy, such that $n_{1}<n_{2}$ if $n_{1}$ is a descendant of
$n_{2}$ in the taxonomy. If it is possible to find a name $n_{max}$
that is the unique maximal element of $N$, then it is permissible
to 
\begin{enumerate}
\item create a monotypic parent $p(x)$ of $x$, and assign $n_{max}$ to
$p(x)$
\item continue handling name clashes at $x$ with the set of possible names
reduced to $N-n_{max}$.
\end{enumerate}
However, its certainly possible that there might not be any such $N_{max}$,
in which case we could just choose a name for $x$ from $N$ (perhaps
not an \emph{incertae sedis} name) and then record all the other names
as equivalents somewhere.

\textbf{BDR:}\textbf{\emph{ }}\emph{we might get this behavior in
a nice an automatic way if we create a single fake leaf for each monotypic
taxonomy node that holds the node's leaf label.}

\subsubsection{Caveats}

When multiple I.S. taxa have been moved to the root node of a subproblem,
they may be I.S. over the entire subproblem, and some may be I.S.
over others in an asymmetric manner. Therefore, we might need to specify
additional information about the original attachment location of the
I.S. taxa, such as their depth. This only affects problems that have
been decomposed.

\textbf{BDR:} \emph{currently we don't actually move taxa to get them
into a subproblem. So, is this even an issue?}

\subsection{Grafted supertree}

\emph{Question:} Does the synthesis tree contain any \emph{incertae
sedis} groups?\\
\emph{Answer:} The grafted supertree will not contain any \emph{incertae
sedis} groups. However, when we attach pruned nodes to a parent in
the grafted supertree, we could mark such nodes \emph{incertae sedis}
if we want.

\subsection{Unpruning}

Currently the unpruner \emph{does not} require that the OTT ids are
named in the grafted solution before unpruning starts. According to
Mark's document, he wasn't sure if such names were generated for nodes
that had an IS taxon placed inside of them, so otc-unprune-solution-and-name-unnamed
nodes throws away all the names and generates them itself.

\textbf{BDR}: See document \texttt{otcetera/doc/unprune-solution-and-name-unnamed-nodes.pdf}

The unpruner should record when unpruned nodes are \emph{incertae
sedis}. Such nodes are unaffected by phylogenies, and so \emph{incertae
sedis} annotations for them make good sense.

\subsection{Annotation}

Annotation primarily involves running a conflict analysis between
the synthesis tree and each input tree. Since neither tree has any
\emph{incertae sedis} taxa, the conflict algorithm does not need to
change. Furthermore, if we allow \emph{incertae sedis} taxa that are
taxonomy-only to be annotated as \emph{incertae sedis} on the synth
tree, then such groups will not affect conflict with the input trees.
We would also like to allow running a conflict analysis between the
synthesis tree and the taxonomy tree. However, naming the nodes \emph{is}
a (almost) run of conflict analysis on the taxonomy tree, and this
has already been done in a prior step. So, the current annotation
procedure actually works as-is.

It would be nice to allow running conflict against the cleaned taxonomy,
though. One way to do this would be to generated a ``placed taxonomy'',
with groups extended to include \emph{incertae sedis} taxa that have
been placed within them. This would not require any updating to the
conflict-analysis code in the annotation step.

\subsection{Conflict service}

The current conflict service considers a group $A$ to conflict with
the taxonomy if group $A$ has an incertae sedis group $B$ placed
within it. This doesn't affect the annotations, since taxon names
are added by the unpruner. But it could make perfectly fine input
trees incorrectly look like they are the cause of broken taxa, if
they contain IS taxa. Thus, it would be nice to have a modified conflict
algorithm. 

\subsubsection{Current conflict algorithm}

The current conflict algorithm is pretty fast, but it works by classifying
tips into either (i) the include group or (ii) the exclude group.
To avoid counting the exclude group for every split, we instead count
the total number of children for each node, and assume that any children
not in the include set are in the exclude set. This is no longer true
when we have incertae sedis taxa. I suspect that if we want to handle
incertae sedies, we'd need a third category (iii) for ``neither include
group nor exclude group''.
\begin{enumerate}
\item Get induced trees on intersection of leaf sets
\item Compute depth for each node (nd->depth)
\item Compute number of tips at or below each node (nd->n\_tips)
\item for each input tree node -> $nd$
\begin{enumerate}
\item skip the root
\item skip monotypic
\item if its a tip then find corresponding (``terminal'') edges in synth
tree and continue
\item leaves1 <- get the list of leaves in the include group of $nd$ (in
input)
\item L2 <- find the total number of tips (L2 = sum {[}nd->n\_tips| nd <-
leaves1{]})
\item leaves2 <- get list of corresponding synth leaf nodes (in synth)
\item nodes <- find all nodes between leaves2 and the MRCA (in synth)
\item MRCA <- mrca of leaves2 (in synth. this uses the nd->depth annotation)
\item Compute number of tips in the include set (nd->include\_tips) below
each node in $nodes$ (in synth)
\item if n\_include\_tips(MRCA) == n\_tips(MRCA) then the MRCA displays
$nd$
\item if n\_include\_tips(MRCA) < n\_tips(MRCA) then
\begin{itemize}
\item foreach node in nodes
\begin{itemize}
\item if (n\_include\_tipes(nd) < n\_tips(nd) and n\_include\_tips(nd) <
l2)
\begin{itemize}
\item this is a conflict!
\end{itemize}
\end{itemize}
\item if there are no conflicts, then this is a resolved\_by.
\end{itemize}
\end{enumerate}
\end{enumerate}

\subsubsection{Modified conflict algorithm?}

This probably is outside the scope of the paper, but if we could come
up with a modified conflict algorithm, that would be nice/useful,
and probably novel. It would also probably be slower...

\section{Results}

Should we do this? We could say: 
\begin{itemize}
\item we placed $x$ incertae sedis taxa.
\item we avoiding breaking $y$ taxa that had IS taxa placed inside them.
\item we allowed $z_{1}$ new taxa into the synthesis tree that were incertae
sedis.
\item we allowed $z_{2}$ new taxa into the synthesis tree that are marked
as extinct.
\item some nodes have as many as $w$\emph{ incertae sedis} children, making
them unbrowseable when incertae sedis children are not excluded.
\item $v_{1}$ input trees w were previously excluded \emph{entirely} because
they are nested within in an incertae sedis taxon.
\item $v_{2}$ input trees w were previously excluded \emph{partially} because
they are nested within in an incertae sedis taxon.
\end{itemize}
Currently the numbers $z_{1}$ and $z_{2}$. 

\section{Discussion}

One thing we could do (perhaps) that we are not currently doing, is
to have nodes marked as incertae sedis on the synth tree. This would
be easy enough if such nodes are not affected in any way by the input
trees. Thus, when unpruning nodes we could mark any nodes \emph{incertae
sedis} if they were marked \emph{incertae sedis} on the taxonomy.

Secondly, I think we need to distinguish \emph{incertae sedis} taxa
that are ``unplaced'' from \emph{incertae sedis} taxa that do not
occur in any input tree. I think that if $A$ contains child $A_{1}$
that is an input tree, and the taxon $A$ is not broken, then $A$
will be placed, and thus any other children $A_{2},A_{3},\ldots,A_{n}$
will also be placed, since they will be added as children of the (placed)
node $A$ by the unpruner. This could be considered when deciding
which nodes to suppress in the tree viewer.

\bibliographystyle{upmplainnat}
\bibliography{otcetera}

\end{document}
